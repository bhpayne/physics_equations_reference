\documentclass[12pt]{article}  %,twocolumn is too scrunched. decrease margin, increase column seperation

%%% 
%%% Composition of formulas for MST Qualifyin Exam
%%% All mistakes are mine, all work is cited by Author
%%% Creative Commons:
%%% Attribution-Noncommercial-Share Alike 3.0 Unported
%%% http://creativecommons.org/licenses/by-nc-sa/3.0/
%%% 

\usepackage{verbatim} % multi-line comments
\usepackage[pdftitle={equations for the qualifying exam}, pdfauthor={Ben Payne}, pdfsubject={equations}, pdfkeywords={maxwell, lagrange}]{hyperref}
\usepackage{times}

% AMS packages and font files
\usepackage{amsmath} % necessary for gathered equations
\usepackage{amsfonts}
%% For figures
%\usepackage[dvipdfm,colorlinks=true]{hyperref}
\usepackage[dvips]{graphicx}
\usepackage[usenames,dvipsnames]{color}
\usepackage{setspace}

% screws up the paragraph formatting
%\raggedright

% margins must be 1 inch:
%\begin{comment}
\setlength{\topmargin}{-.5in}
\setlength{\textheight}{9.5in}
\setlength{\oddsidemargin}{0in}
\setlength{\textwidth}{6.5in}
%\end{comment}

%% Commonly used macros
% from https://facetsproject.org/facets/browser/trunk/docs/gsdocs/nondimtf2hall.tex
\newcommand{\pfrac}[2]{\frac{\partial #1}{\partial #2}}
\newcommand{\pfraca}[1]{\frac{\partial}{\partial #1}}
\newcommand{\pfracb}[2]{\partial #1/\partial #2}

\begin{document}
%\twocolumn
\title{Physics equations}
\author{Ben~Payne\footnote{Electronic address: ben.is.located@gmail.com}}
%\affiliation{Department~of~Physics, Missouri~University~of~Science~\&~Technology, Rolla,~MO~65409}
\date{\today}

\thispagestyle{empty}  % hides page number of the title page
\pagestyle{empty} % no page numbers on the following pages
%\begin{abstract}
%equations for the qualifying exam
%\end{abstract}

version 0.03, 20090512

%%%%%%%%%%%%%%%%%%%%%%%%%%%%%%%%%%%%%%%%%%%%%%%%%%%%%%%%%%%%%%%%%%%%%%%%%%%%%
\section{unit conversion}
%%%%%%%%%%%%%%%%%%%%%%%%%%%%%%%%%%%%%%%%%%%%%%%%%%%%%%%%%%%%%%%%%%%%%%%%%%%%%

%see http://en.wikibooks.org/wiki/LaTeX/Tables
%for help with tables

see http://physics.nist.gov/cuu/Units/units.html and \cite{ZemanskyYoungGen} page T-5 \\

\begin{tabular}{lcccc}
Description                    & Name    & Symbol   & convert          & SI \\
\hline
\\[0.1pt]
Force                          & Newton  & N        &                   & $\frac{m \ kg}{s^2}$  \\
\\[0.1pt]
energy, work, quantity of heat & joule   & J        & $N \ m$            & $\frac{m^2 \ kg}{s^2}$ \\
\\[0.1pt]
power                          & Watt    & W        & $\frac{J}{s}$      & $\frac{m^2 \ kg}{s^3}$ \\
\\[0.1pt]
pressure, stress               & pascal  & Pa       & $\frac{N}{m^2}$    & $\frac{kg}{m \ s^2}$ \\
\\[0.1pt]
electric charge, 
quantity of electricity        & coulomb & C        &                   & $s \ A$ \\ % amp/second?
\\[0.1pt]
electric potential difference,
electromotive force            & volt    & V        & W/A  % J/C?           & $\frac{m^2 \ kg}{s^3 \ A}$ \\
\\[0.1pt]
capacitance                    & farad   & F        & C/V             & $\frac{s^4 \ A^2}{m^2 \ kg}$ \\
\\[0.1pt]
electric resistance            & ohm     & $\Omega$ & V/A             & $\frac{m^2 \ kg}{s^3 \ A^2}$ \\
\\[0.1pt]
magnetic field                 & Tesla   & T        & $\frac{V \ s}{m}$  % N/(A m)? & $\frac{m \ kg}{s^2 \ A}$ \\
\end{tabular}

\ \\

$1 \text{Joule} = 6.24150974 \ 10^{18}$ electron volts

\ \\
$k_{Boltzmann} T_{room} = \frac{1}{40}$ eV

\ \\
$T_{room} = 293.15 K$

%%%%%%%%%%%%%%%%%%%%%%%%%%%%%%%%%%%%%%%%%%%%%%%%%%%%%%%%%%%%%%%%%%%%%%%%%%%%%
\section{constants}
%%%%%%%%%%%%%%%%%%%%%%%%%%%%%%%%%%%%%%%%%%%%%%%%%%%%%%%%%%%%%%%%%%%%%%%%%%%%%

see http://en.wikipedia.org/wiki/Physical\_constant and  http://physics.nist.gov/cuu/Constants/

\ \\

Universal constants : \\

\begin{tabular}{lcr}
Description                       & Symbol                  & SI value \\
\hline
\\[0.1pt]
speed of light in vacuum          & c                       & $299 792 458 \ \frac{m}{s}$ \\
\\[0.1pt]
Newtonian constant of gravitation & G                       & $6.67428(67) \ 10^{−11} \frac{m^3}{kg \ s^2}$ \\
\\[0.1pt]
Planck's constant                 & h                       & $6.626 068 96(33) \ 10^{−34} \  J \ s$ \\
\\[0.1pt]
reduced Planck constant           & $\hbar = \frac{h}{2 \pi}$ & $1.054 571 628(53) \ 10^{−34} \ J \ s$ \\
\end{tabular}

\ \\ 

\begin{tabular}{lcr}
Description                             & Symbol, definition                         & SI value \\
\hline
\\[0.1pt]
magnetic constant (vacuum permeability) & $\mu_0$                                    & $4 \pi \ 10^{−7} \frac{N}{A^2}$ \\
\\[0.1pt]
electric constant (vacuum permittivity) & $\epsilon_0 = \frac{1}{\mu_0 c^2}$         & $8.854 187 817 \ 10^{−12} \ \frac{F}{m}$ \\
\\[0.1pt]
elementary charge                       & $e$                                       & $1.602 176 487(40) \ 10^{−19} \ C$ \\
\\[0.1pt]
Bohr radius                             & $a_0 = \frac{\alpha}{4 \pi R_\infty}$      & $0.529 177 2108(18) \ 10^{−10} \ m$ \\
\\[0.1pt]
classical electron radius               & $r_e = \frac{e^2}{4\pi\epsilon_0 m_e c^2}$ & $2.817 940 2894(58) \ 10^{−15} \ m$ \\
\\[0.1pt]
electron mass                           & $m_e$                                      & $9.109 382 15(45) \ 10^{−31} \ kg = 511 \frac{KeV}{c^2}$  \\
\\[0.1pt]
fine-structure constant                 &$ \alpha = \frac{\mu_0 e^2 c}{2 h} = 
                                          \frac{e^2}{4 \pi \epsilon_0 \hbar c}$     & $7.297 352 5376(50) \ 10^{−3}$ \\
proton mass                             & $m_p $                                     & $1.672 621 637(83) \ 10^{−27} \ kg$ \\
\\[0.1pt]
Rydberg constant                        & $R_\infty = \frac{\alpha^2 m_e c}{2 h}  $        & $10 973 731.568 525(73) \ \frac{1}{m}$ \\
\\[0.1pt]
atomic mass unit 
\\[0.1pt]
(unified atomic mass unit)              & $m_u = \frac{1}{u} $                             & $1.660 538 86(28) \ 10^{−27} \ kg$ \\
\\[0.1pt]
Avogadro's number                       & $N_{Av}$                                   & $6.022 1415(10) \ 10^{23} \ \frac{1}{mol}$ \\
\\[0.1pt]
Boltzmann constant                      & $k_{B} = \frac{R}{N_{Av}} $                         & $1.380 6505(24) \ 10^{−23} \ \frac{J}{K}$ \\
\\[0.1pt]
gas constant                            & $R$                                        & $8.314 472(15) \ \frac{J}{K \ mol}$ \\
\\[0.1pt]
standard atmosphere                     & $atm$                                      & $101 325 \ Pa = 14.7 \ Psi $
\end{tabular}
\ \\
Miscellaneous: 
\begin{equation}
 1 \ eV = 1.602 \ 10^{-19} J
\end{equation}

standard temperature is $293.15 \ K$

\begin{equation}
 \frac{m_{proton}}{m_e} = 1836
\end{equation}


%%%%%%%%%%%%%%%%%%%%%%%%%%%%%%%%%%%%%%%%%%%%%%%%%%%%%%%%%
\section{Symbol notations}
%%%%%%%%%%%%%%%%%%%%%%%%%%%%%%%%%%%%%%%%%%%%%%%%%%%%%%%%%

See page 631-642 of \cite{ReifThermo}

$C_v \equiv$ specific heat for constant volume

$k_b \equiv$ Boltzmann constant

$\vec{E} \equiv$ electric field

$\vec{B} \equiv$ magnetic field

$\vec{D} \equiv$ electric displacement, equ 4.21, page 175 \cite{GriffithED}

$\vec{N} \equiv$ torque

$W \equiv$ work

$\vec{v} \equiv$ velocity

$V \equiv$ electrostatic potential [Volts]

$V \equiv$ volume

$U \equiv$ potential energy [Volts]

$\vec{P} \equiv$ polarization, page 166 \cite{GriffithED}

$I \equiv$ current [Amps]

$\vec{F} \equiv$ force [Newtons, $\frac{kg \cdot m}{s^2}$]

$E \equiv$ energy

$T \equiv$ kinetic energy

$T \equiv$ temperature

$L \equiv$ lagrangian

$L \equiv$ capacitance

$\vec{L} \equiv$ classical (orbital) angular momentum

$l \equiv$ quantum (orbital) angular momentum

$\vec{p} \equiv$ linear momentum

$\omega \equiv$ angular frequency

$z \equiv$ single particle partition function

$Z \equiv$ total partition function

%%%%%%%%%%%%%%%%%%%%%%%%%%%%%%%%%%%%%%%%%%%%%%%%%%%%%
\begin{thebibliography}{99}

\bibitem{ZemanskyYoungGen}
Sears, Zemansky, Young \textit{University Physics, Fifth edition}, (1976)

\bibitem{GriffithED}
Griffith \textit{Intro to Electrodynamics}, (1999)

\bibitem{JacksonED}
Jackson \textit{Classical Electrodynamics}, (1999)

\bibitem{GoldsteinCM}
Goldstein \textit{Classical Mechanics}, (1980) (second edition)\\
Available from Aaron

\bibitem{ReifThermo}
Reif \textit{Fundamentals of statistical and thermal physics}, (1965)

\bibitem{ParrisQM}
Parris's book on \textit{Quantum mechanics}, (2008) \\
\htmladdnormallink{463 course page}{http://physics.mst.edu/classes/class_463notes.html}

\bibitem{LiboffQM}
Liboff \textit{Introduction to Quantum mechanics}, (2003) Fourth edition

\bibitem{MarionCM}
Marion \textit{Classical Dynamics of Particles and systems}, (1970) Second edition. \\
Available from Prof Waddill, Elizabeth

\bibitem{TiplerMP}
Tipler, Llewellyn \textit{Modern Physics}, (1999)

\end{thebibliography}

\end{document}