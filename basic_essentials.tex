\documentclass[12pt]{report} 
%\usepackage[pdftex]{hyperref}
\usepackage{amsmath} % advanced math
\usepackage{amssymb}
\usepackage{ulem}
\usepackage{graphicx,color}
%\usepackage{subfigure}
\usepackage{verbatim} % multi-line comments
\usepackage[backref, colorlinks=false, pdftitle={20110125}, 
pdfauthor={Ben Payne, Alexey Yamilov}, pdfsubject={meeting}, 
pdfkeywords={localization, gain, transmission, random, media}]{hyperref}
%\usepackage{hyperref} % hyper links
%I'd like to use "backpageref" instead of linking back to section numbers
\setlength{\topmargin}{-.5in}
\setlength{\textheight}{9in}
\setlength{\oddsidemargin}{0in}
\setlength{\textwidth}{6.5in}
\newcounter{fignum}
\newcommand{\fignum}{\stepcounter{fignum}\arabic{fignum}}
\begin{document}
\section{Gaussian and SI}
\begin{tabular}{c c | c c}
Gaussian (``CGS'') & & SI & \\
cm, gram, second & & m, kg, second &\\\hline
&&&\\
$\vec{\nabla} \times \vec{E} = -\frac{1}{c} \frac{\partial \vec{B}}{\partial t}$ & $\vec{\nabla} \times \vec{H} = \frac{1}{c} \frac{\partial \vec{D}}{\partial t} + \frac{4\pi}{c}\vec{J}$ & $\vec{\nabla} \times \vec{E} = -\frac{1}{c} \frac{\partial \vec{B}}{\partial t}$ & $\vec{\nabla} \times \vec{H} = \frac{\partial \vec{D}}{\partial t} + \vec{J}$ \\
&&&\\
$\vec{\nabla} \cdot \vec{D} = 4 \pi \rho $ & $\vec{\nabla} \cdot \vec{B}=0$ & $\vec{\nabla} \cdot \vec{D} = \rho $ & $\vec{\nabla} \cdot \vec{B}=0$ \\
&&&\\\hline
&&&\\
$\vec{F} = q \left(\vec{E} + \frac{1}{c}\vec{v}\times \vec{B}\right)$ & & $\vec{F} = q \left(\vec{E} + \vec{v}\times \vec{B}\right)$ & \\ 
&&&\\\hline
&&&\\
$\vec{D} \equiv \vec{E} + 4\pi \vec{P}$ & $\vec{H} \equiv \vec{B}-4\pi \vec{\mu}$ & $\vec{D} \equiv \epsilon_0\ \vec{E} + \vec{P}$ & $ \vec{H} = \frac{1}{\mu_0} \vec{B} - \vec{\mu}$ \\
\end{tabular}

\ \\

Conversion factors between Gaussian, SI. Usage: $X_{Gaussian} = k_X X_{SI}$

\ \\
\begin{tabular}{c c c}
$k_{\vec{D}} = \sqrt{4\pi/\epsilon_0}$ & $k_{\vec{H}} = \sqrt{4 \pi \mu_0}$ & $k_{\rho} = k_{\vec{J}} = k_{\vec{P}} = 1/\sqrt{4 \pi \epsilon_0}$\\
&& \\
$k_{\vec{E}} = \sqrt{4 \pi \epsilon_0}$ & $k_{\vec{B}} = \sqrt{4 \pi /\mu_0}$ & $k_{\mu} = \sqrt{\mu_0/(4 \pi)}$ \\
\end{tabular}

\ \\

For dimensions, $k_{length} = 100$. $k_{mass}=1000$.
\ \\


\begin{equation}
 \epsilon_0 \mu_0 = \frac{1}{c^2}
\end{equation}
\begin{equation}
 \mu_0 = 4 \pi \cdot 10^{-7}
\end{equation}

\section{basics}
\begin{equation}
 E = h\ f = \frac{h\ c}{\lambda}  
\end{equation}
\begin{equation}
 F = -G \frac{m_1\ m_2}{r^2}
\end{equation}
Explain the following momentum relation physically
\begin{equation}
 p = \hbar k =mv
\end{equation}
Explain the following relation between flux and velocity physically
\begin{equation}
 \vec{J} = n \vec{v}
\end{equation}
\begin{equation}
 KE = \frac{1}{2} m v^2 \quad \quad \quad PE = mgh
\end{equation}



\end{document}
