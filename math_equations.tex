\documentclass[12pt]{article}  %,twocolumn is too scrunched. decrease margin, increase column seperation

%%% 
%%% Composition of formulas for MST Qualifyin Exam
%%% All mistakes are mine, all work is cited by Author
%%% Creative Commons:
%%% Attribution-Noncommercial-Share Alike 3.0 Unported
%%% http://creativecommons.org/licenses/by-nc-sa/3.0/
%%% 

\usepackage{verbatim} % multi-line comments
\usepackage[pdftitle={equations for the qualifying exam}, pdfauthor={Ben Payne}, pdfsubject={equations}, pdfkeywords={maxwell, lagrange}]{hyperref}
\usepackage{times}

% AMS packages and font files
\usepackage{amsmath} % necessary for gathered equations
\usepackage{amsfonts}
%% For figures
%\usepackage[dvipdfm,colorlinks=true]{hyperref}
\usepackage[dvips]{graphicx}
\usepackage[usenames,dvipsnames]{color}
\usepackage{setspace}

% screws up the paragraph formatting
%\raggedright

% margins must be 1 inch:
%\begin{comment}
\setlength{\topmargin}{-.5in}
\setlength{\textheight}{9.5in}
\setlength{\oddsidemargin}{0in}
\setlength{\textwidth}{6.5in}
%\end{comment}

%% Commonly used macros
% from https://facetsproject.org/facets/browser/trunk/docs/gsdocs/nondimtf2hall.tex
\newcommand{\pfrac}[2]{\frac{\partial #1}{\partial #2}}
\newcommand{\secpfrac}[2]{\frac{\partial ^2 #1}{\partial #2 ^2}}
\newcommand{\pfraca}[1]{\frac{\partial}{\partial #1}}
\newcommand{\secpfraca}[1]{\frac{\partial}{\partial #1}}
\newcommand{\pfracb}[2]{\partial #1/\partial #2}

\begin{document}
%\twocolumn
\title{Physics equations}
\author{Ben~Payne\footnote{Electronic address: ben.is.located@gmail.com}}
%\affiliation{Department~of~Physics, Missouri~University~of~Science~\&~Technology, Rolla,~MO~65409}
\date{\today}

\thispagestyle{empty}  % hides page number of the title page
\pagestyle{empty} % no page numbers on the following pages
%\begin{abstract}
%equations for the qualifying exam
%\end{abstract}

version 0.01, 20090408


"del" operator $\nabla$ has different names, depending on how it is applied.

gradient: $\nabla f$

divergence: $\nabla \cdot f$

curl: $\nabla \times f$

\subsection{math 402}

\begin{equation}
 e^x = \sum_{n=0}^{\infty} \frac{x^n}{n!}
\end{equation}

\begin{equation}
 f(\vec{r}_0 + \Delta \vec{r})=e^{\Delta \vec{r} \cdot \vec{\nabla}} f(\vec{r}) | _{\vec{r}=\vec{r}_0} = 
\sum_{m=0}^{\infty} \frac{(\Delta \vec{r} \cdot \vec{\nabla})^m}{m!} f(\vec{r}) | _{\vec{r}=\vec{r}_0}
\end{equation}

Einstein summation notation (relies on covarient versus contravarient notation)
\begin{equation}
 \vec{A} \times \vec{B} = \epsilon^{ijk} \hat{x}_i A_j B_k
\end{equation}

\begin{equation}
 \epsilon^{lmn} \epsilon_{ljk} = \delta_{mj} \delta_{nk} - \delta_{mk} \delta_{nj}
\end{equation}

Dr Hale's favorite formula:
\begin{equation}
 df = d\vec{r} \cdot \vec{\nabla}f
\end{equation}

\subsection{partial differential equations}

\begin{equation}
 \nabla^2 = \secpfraca{\rho} + \frac{1}{\rho} \pfraca{\rho} + \frac{1}{\rho^2} \secpfraca{\phi} + \secpfraca{z}
\end{equation}



most are given in 2D, but are applicable to nD 

Laplace's equation
\begin{equation}
 \secpfrac{u}{x} + \secpfrac{u}{y} =0
\end{equation}
more simply for n dimensions
\begin{equation}
 \nabla^2 u = 0
\end{equation}
Wave equation
\begin{equation}
 \secpfrac{u}{x} - \secpfrac{u}{y} = 0
\end{equation}
Heat equation
\begin{equation}
 \secpfrac{u}{x} - \pfrac{u}{y} = 0
\end{equation}
Poisson equation
\begin{equation}
 \secpfrac{u}{x} + \secpfrac{u}{y} = g(x,y)
\end{equation}
more simply for n dimensions
\begin{equation}
 \nabla^2 u = g(x,y)
\end{equation}


\end{document}
